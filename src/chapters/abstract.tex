This thesis presents a an adaptive augmented, gain-scheduled baseline LQR-PI controller applied to the Road Runner six-degree-of-freedom generic hypersonic vehicle model.
Uncertainty in control effectiveness, longitudinal center of gravity location, and aerodynamic coefficients are introduced in the model, as well as sensor bias and noise, and input time delays.
The performance of the baseline controller is compared to the same design augmented with one of two different model-reference adaptive controllers: a classical open-loop reference model design, and modified closed-loop reference model design.
Both adaptive controllers show improved command tracking and stability over the baseline controller when subject to these uncertainties.
The closed-loop reference model controller offers the best performance, tolerating a reduced control effectiveness of 50\%, rearward center of gravity shift of -0.9 to -1.6 feet (6--11\% of vehicle length), aerodynamic coefficient uncertainty scaled $4\times$ the nominal value, and sensor bias of $+1.6$ degrees on sideslip angle measurement.
The closed-loop reference model adaptive controller maintains at least 73\% of the delay margin provided by the robust baseline design, tolerating input time delays of between 18--46 ms during 3 degree angle of attack doublet, and 80 degree roll step commands.
